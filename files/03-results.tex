\chapter{Ergebnisse und Diskussion} \label{chap:2}
\section{Zusammensetzung der Kampagnen}

\section{Erste Ergebnisse und Vergleiche}
\subsection{Vergleich der Oberflächentemperatur des Divertors in OP1.2b und OP2 und Einfluss auf Erosionsprozesse}
Den größten Unterschied zwischen OP1.2b und OP2 in Hinsicht auf die Erosionsprozesse in W7-X stellt der Austausch des Divertors dar. Der Wechsel von ungekühlten, feinkörnigen Graphitkacheln zu wassergekühlten HHF-CFC Divertortargets \cite{MarkusOP1.2a} bringt eine Änderung der Materialeigenschaften ebenso wie eine deutlich geringere Oberflächentemperatur mit sich. Letztere ist auf das Abführen von Wärme durch die tieferliegenden Wasserleitungen zurückzuführen. Hatte sich der Divertor in OP1.2b über den Tag noch deutlich aufgeheizt (von einer Durchschnittstemperatur der TM2h und TM3h von ca. \qty{100}{\degreeCelsius} auf über \qty{200}{\degreeCelsius}) und teilweise lokale Werte jenseits von \qty{700}{\degreeCelsius} erreicht, ist die Temperatur in OP2 zu allen Tageszeiten ähnlich (Durchschnitt von \qty{25}{\degreeCelsius} bis \qty{40}{\degreeCelsius} auf TM2h und TM3h). Lokale Maxima in OP2 weisen Temperaturen um \qty{220}{\degreeCelsius} auf. Wird der Bereich der Strikeline betrachet, liegt der Standardwert der Oberflächentemperatur bei \qty{550}{\degreeCelsius} in OP1.2b und \qty{150}{\degreeCelsius} in OP2. Grob gesagt kann von einer Viertelung der Maximaltemperatur gesprochen werden \cite{divertorT}.\\

Werden die obigen Temperaturen und Standardwerte für Elektronendichte und -temperatur in die Formeln für Zerstäbungsausbeuten und Erosionsraten eingesetzt, so zeigt sich ein geringer Einfluss des Temperaturunterschieds. Unter Nutzung von \(\alpha\) = \qty{40}{\degree}, \(n_e\) = \qty[per-mode = symbol]{2.302e19}{\per\m\tothe{3}} und \(T_e\) = \(T_i\) = \qty{13.26}{eV} und der Formeln \ref{eq:YphysEck}, \ref{eq:flux}, \ref{eq:YchemRoth}, \ref{eq:totalY}, \ref{eq:erosionRate}, \ref{eq:erosionLayer} ergibt sich für \(T_{s, OP1}\) = \qty{550}{\degreeCelsius} und \(T_{s, OP2}\) = \qty{150}{\degreeCelsius} eine Zerstäubungsausbeute für Wasserstoff von \(Y_{H, OP1}\) = 0.00390 für die niedrige Oberflächentemperatur und \(Y_{H, OP2}\) = 0.00389 für die hohe. Die Zerstäubungsausbeuten von Kohlenstoff und Sauerstoff sind temperaturunabhängig und haben in beiden Beispielen Werte von \(Y_{C}\) = 0.12534 beziehungsweise \(Y_{O}\) = 0.73538. Mit diesen repräsentative Parametern ergibt sich für OP1.2b eine Erosionsrate von \qty[per-mode = symbol]{6.948e-8}{\m\per\s}, für OP2 liegt sie bei \qty[per-mode = symbol]{6.947e-8}{\m\per\s}. Eine exemplarische Entladung über \(t_{discharge}\) = \qty{10}{s} würde also eine Schichtdicke von \qty[per-mode = symbol]{694.8}{\nm} oder \qty[per-mode = symbol]{694.7}{\nm} verlieren. Zu betonen ist, dass dieses Ergebnis die Bruttoerosion beschreibt. Wiederablagerung wurde in diesem Fall nicht betrachtet.\\

Der Vergleich des Ergebnisses mit einem Rechenbeispiel aus \cite{PWI-Dirk} zeigt eine ähnliche Größenordnung. Dort ist für \(n_e\) = \qty[per-mode = symbol]{2.5e20}{\per\m\tothe{3}}, \(T_e\) = \(T_i\) = \qty{10}{eV}, einen Ionenbeschuss mit Deuterium und einer Gesamtausbeute von \(Y\) = 0.01 eine Schicht von \qty{0.7}{mm} abgetragen worden. Die Laufzeit der Entladung ist auf \(t_{discharge}\) = \qty{1000}{s} gesetzt worden, sodass die zugehörige Erosionsrate \qty[per-mode = symbol]{7e-7}{\m\per\s} beträgt. Der Unterschied von einer Größenordnung ist auf die verschiedenen Elektronendichten zurückzuführen. Wird \(n_e\) = \qty[per-mode = symbol]{2.302e20}{\per\m\tothe{3}} eingesetzt, so ist die Erosionsrate \qty[per-mode = symbol]{6.692e-7}{\m\per\s} in OP1.2b, für OP2 hat sie den Wert \qty[per-mode = symbol]{6.693e-7}{\m\per\s}.\\

Interessant ist hierbei, dass eine Erhöhung der Elektronendichte um eine Größenordnung zu einer Umkehr der Erosionsraten führt. War zunächst die Abtragung am heißen Target höher, ist bei \(n_e\) = \qty[per-mode = symbol]{2.302e20}{\per\m\tothe{3}} die des neuen Divertortargets größer. WAS SCHLUSSFOLGERN WIR DARAUS?

\subsection{Erosionsraten in OP1.2}
Experimentelle Bestimmungen der Ersoionsraten in OP1.2a, OP1.2b und OP2.1 ergaben, dass die stärkste Nettoerosion im Bereich der Strikeline stattfindet. Netto-Wiederablagerung wurde in unmittelbarer Nähe der Strikeline, aber auch an den anderen PFC Oberflächen festgestellt. Dabei wurde eine gewisse Asymmetrie zwischen den Divertoreinheiten beobachtet, sowohl bei toroidal symmetrischen Einheiten miteinander, als auch beim Vergleich der unteren und oberen Einheit eines Moduls. Diese folgt der Asymmetrie der Oberflächentemperaturen der Divertortargets \cite{MarkusOP1.2a, MarkusOP1.2b, erosionOP12}.\\

Während die Erosion sich qualitativ zwischen den verschiedenen Kampagnen wenig unterscheidet, ist der quantitative Unterschied in den Erosionsraten zwischen OP1.2a und OP1.2b sehr ausgeprägt. In OP1.2a ist noch eine Nettoerosion von \qtyrange[per-mode=symbol]{5.8}{8.4}{\nm\per\s} aufgetreten, hatte dieser Wert sich in OP1.2b auf \qtyrange[per-mode=symbol]{1.1}{2.5}{\nm\per\s} reduziert. Das wird mit dem Einführen regelmäßiger Borierungen begründet, wodurch die Konzentration von Sauerstoff- und Kohlenstoffionen stark verringert wurde \cite{MarkusOP1.2a, MarkusOP1.2b, erosionOP12}.\\

Bei der Betrachtung der erodierten Oberflächen ist auffällig, dass diese in ihrer Struktur glatter sind als nicht erodierte Bereiche des Divertors. Das hat Einfluss auf den Einfallswinkel von Ionen und damit auf die Erosion, welche auf glatten Oberflächen höher ist \cite{MarkusOP1.2a, MarkusOP1.2b}.
