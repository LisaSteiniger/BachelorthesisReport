\chapter{Einleitung in die Plasmaphysik an Wendelstein 7-X} \label{chap:1}
%Look-Ups:\\
%Divertor und Targetmodule, technische Details \cite{TM2013}\\
%(Technische) Details W7-X und Infos zu OP2 \cite{W7XtechDetailsOP2}\\
%Allgemeines zu W7-X über die Kampagnen, Standardgrößen für heizleistung etc., allgemeines zu Divertor und PFCs in Literature/Dirk\_DivertorDetails...\\

Fusion bezeichnet das Verschmelzen von zwei Atomkernen zu einem schweren Atomkern. Handelt es sich um leichte Atomkerne, wie bei der Fusion der Wasserstoffisotope Deuterium und Tritium zu Helium, wird dabei Energie freigesetzt. Der Grund dafür liegt in der Umwandlung eines Teils der Masse in Energie nach der Formel \(E = m c^2\), da die Edukte in Summe schwerer sind als die Produkte der Fusionsreaktion. Deshalb strahlen beispielsweise Sterne Energie ab. Allerdings sind für den Ablauf von Fusionsreaktionen extreme Bedingungen notwendig, um die Abstoßung zwischen den positiv geladenen Atomkernen zu überwinden. Die dafür nötige Energie erhalten die Atomkerne durch hohe Temperaturen, wobei hohe Dichten und lange Einschlusszeiten die Wahrscheinlichkeit für den Ablauf einer Verschmelzung erhöhen. Während diese Zustände in Sternen jedoch einfach erreicht werden, indem ihre eigene Masse die Teilchen durch Gravitationskräfte zusammenhält, gestaltet sich das Erreichen fusionsrelevanter Umgebungsbedingungen in Fusionskraftwerken auf der Erde schwieriger. Dennoch wird aufwendige Forschung betrieben, um dieses Problem zu lösen \cite{IntroStellarator}.\\

Fusionskraftwerke haben den Vorteil, sicher im Betrieb zu sein und kein Kohlenstoffdioxid oder andere klimaschädliche Gase freizusetzen. Ein weiterer positiver Aspekt besteht in der Unabhängigkeit von seltenen Ressourcen, da die Treibstoffe weltweit verfügbar sind. Deuterium ist im Meerwasser enthalten, während Tritium zum Beispiel durch die Bestrahlung von Tritium mit schnellen Neutronen gewonnen werden kann \cite{IntroStellarator}. Außerdem wird durch Fusion kein stark radioaktiver Abfall produziert, wie es bei Kernkraftwerken der Fall ist.\\

Trotzdem ist dazu anzumerken, dass durch den Kontakt mit schnellen Teilchen Material im Reaktor verstrahlt wird. Weiterhin ist Tritium als Treibstoff der Fusionsreaktion radioaktiv \cite{FusionIAEA}. Die Halbwertszeiten der in Fusionsrektoren auftretenden radioaktiven Isotope liegen jedoch nur in der Größenordnung einiger Jahre \cite{FusionIAEA, IntroStellarator}. Es soll an dieser Stelle ebenfalls noch erwähnt werden, dass die ausreichende Verfügbarkeit von Ressourcen nicht für die Materialien gilt, aus denen der Reaktor selbst besteht. Die Produktionskapazitäten für supraleitende Kabel sind beispielsweise knapp \cite{PWI-Dirk}.\\

Dennoch machen die genannten Gründe Fusionskraftwerke insgesamt zu einer attraktiven Energiequelle der Zukunft. Bis sie weit genug entwickelt sind, um dieser Aufgabe nachzukommen, sind aber noch viele Fragen zu beantworten. Ihrer Klärung widmet sich die Plasmaphysik, indem sie das Verhalten von Plasmen und mögliche Reaktorkonzepte untersucht.\\   

Um einen Reaktor zu betreiben, ist es zunächst notwendig, im vakuumierten Plasmagefäß ein entsprechendes Plasma zu erzeugen und einzuschließen \cite{FusionIAEA}. Dabei stellt sich die Herausforderung, dass Plasmen auf der Erde \qty{100e6}{\degreeCelsius} \cite{FusionIAEA} erreichen müssen, damit es zur Fusion kommt. Diesen Temperaturen hält allerdings kein Material stand. Akzeptable thermische Lasten liegen nur im Bereich von \qty[per-mode=symbol]{10}{\MW\per\m\tothe{2}} \cite{PWI-Dirk}, der Plasmaeinschluss muss demnach anders gestaltet werden. Dazu gibt es mehrere Konzepte, unter anderem Inertialeinschluss und Einschluss durch ein magnetisches Feld \cite{IntroStellarator}. Letztere Möglichkeit soll hier genauer erläutert werden.\\

Der Einschluss von Plasma durch ein Magnetfeld beruht auf der Eigenschaft des Plasmas, aus geladenen Teilchen zu bestehen. Das gilt per Definition für jedes Plasma, da Plasmen dadurch entstehen, dass ein Gas soweit erhitzt wird, dass sich seine Atome in positiv geladene Plasmaionen und Elektronen aufspalten. Die Ladung der Plasmapartikel führt dann dazu, dass elektrische und magnetische Felder Kräfte auf sie ausüben. Im Fall von Magnetfeldern handelt es sich dabei um die Lorentzkraft, die bewirkt, dass Ionen und Elektronen um die Magnetfeldlinie kreisen, entlang derer sie sich bewegen. Diese Bewegung wird Gyration genannt. Der Radius ist masseabhängig, die Richtung wird durch das Vorzeichen der Ladung vorgegeben. Elektronen beschreiben daher enge Spiralen oder Kreise in eine Richtung, Ionen weitere Spiralen oder Kreise in entgegengesetzter Richtung. Dadurch sind die Teilchen an die nähere Umgebung der Magnetfeldlinien gebunden \cite{IntroStellarator}.\\

Um die Plasmapartikel von der Wand des Plasmagefäßes fernzuhalten, muss ein entsprechend geformtes Magnetfeld aufgebaut werden. Dieses ist toroidal geschlossen und poloidal verdrillt, das Maß der Verdrillung wird durch die Kenngröße der Rotationstransformation (iota) ausgedrückt \cite{IntroStellarator}. Die toroidale Feldkomponente garantiert parallelen Einschluss, weil die Teilchen sich entlang der geschlossenen Magnetfeldlinien im Kreis bewegen. Die ploidale Komponente verhindert, dass es zur Ladungsseparation und damit zum nach außen gerichteten Drift der Teilchen kommt. Das wird in \cite{PWI-Dirk, IntroStellarator} näher erläutert. Ein solches Magnetfeld kann auf unterschiedliche Art erzeugt werden, die beiden vielversprechendsten Konzepte sind der Tokamak und der Stellarator \cite{IntroStellarator}.\\

Tokamaks weisen eine einfache Geometrie der Spulen zur Erzeugung des Magnetfelds auf, da diese nur den toroidalen Teil erzeugen müssen. Die Verdrillung der Magnetfeldlinien wird durch einen Plasmastrom erreicht. Dadurch ist das Magnetfeld und der ganze Aufbau eines Tokamaks toroidal symmetrisch, allerdings erlaubt dieses Konzept nur gepulste Operation und ist anfällig für Instabilitäten durch Schwankungen des Plasmastroms. Im Gegensatz dazu ist die Konstruktion eines Stellarators deutlich anspruchsvoller, Spulen und Magnetfeld sind komplizierter geformt und nicht in jedem toroidalen Winkel gleich. Das ermöglicht aber ein vollständig durch externe Spulen erzeugtes Magnetfeld, sodass kein Plasmastrom nötig ist. Somit ist der kontinuierliche Betrieb gewährleistet und auch die Instabilität durch den Plasmastrom fällt weg \cite{IntroStellarator}.\\  

Wendelstein 7-X (W7-X) ist ein vom Max-Planck-Institut für Plasmaphysik Greifswald durchgeführtes Experiment vom Typ Stellarator. Es ist seit 2015 in Betrieb und gehört mit einem durchschnittlichen Innendurchmesser von \qty{0.5}{m} und einem durchschnittlichen Außendurchmesser von \qty{5.5}{m} \cite{W7XtechDetailsOP2} zu den größten und am weitesten entwickelten Anlagen dieser Art \cite{LargestStellarator1, LargestStellarator2, LargestStellarator3}. In Greifswald wird an Wasserstoff- und Heliumplasmen geforscht, Fusionsreaktionen finden nicht statt. 
%(Quelle). 
Derzeit werden vornehmlich Wasserstoffplasmen untersucht bei denen Ionentemperaturen von \qty{2.8}{keV}, Einschlusszeiten von \qty{250}{ms} und linienintegrierte Elektronendichten von \qty[per-mode=symbol]{1.6e20}{\per\m\tothe{3}} erreicht werden \cite{DivertorDirk}. Ziel ist es, Entladungen über \qty{30}{min} bei einer Heizleistung von \qty{10}{MW} zu realisieren \cite{DivertorCooled}. Wendelstein 7-X besteht aus fünf identischen Modulen, die in sich jeweils eine \qty{180}{\degree} Punktsymmetrie aufweisen \cite{W7XtechDetailsOP2}.\\

Das Magnetfeld von W7-X wird durch 50 nicht-planare und 20 planare Spulen erzeugt, Trimspulen erlauben kleinere Anpassungen im laufenden Betrieb \cite{W7XtechDetailsOP2}. Das Magnetfeld spiegelt die fünffaltige toroidale Symmetrie der Anlage wieder und kann in verschiedenen Magnetfeldkonfigurationen auftreten. Die gängigsten sind die Standardkonfiguration (EIM), die High-Iota Konfiguration (FTM) und die High-Mirror Konfiguration (KJM) \cite{W7XtechDetailsOP2}. Je nach Konfiguration hat das eingeschlossene Plasma eine andere Form. Allgemein ist der Aufbau jedoch der folgende:
%(?): 
Im Innersten des Einschlusses befindet sich  das Core Plasma, es ist die heißeste Zone, die Magnetfeldlinien sind in sich geschlossen. Weiter außen folgt das Randplasma, getrennt vom Core Plasma durch die sogenannte Separatrix. Das Randplasma wird durch magnetische Inseln geformt und ist von niedrigerer Temperatur. Die Anzahl und Position der magnetischen Inseln hängt dabei von der Magnetfeldkonfiguration ab wie Abb. \ref{fig:Poincare} zeigt. 
%Der Bereich geschlossener Magnetfeldlinien endet mit der letzten geschlossenen Flussfläche (Last closed flux surface, LCFS) auf die die Scrape Off Layer (SOL) folgt. Innerhalb der SOL sind die Magnetfeldlinien geöffnet
\cite{W7XtechDetailsOP2}.\\

\begin{figure}[!htb] 
	\centering
	\subfigure[Standard (EIM)]{\includegraphics[width=0.32\textwidth]{figures/PoincareEIM_lowiota.png}}
	\subfigure[High-Mirror (KJM)]{\includegraphics[width=0.32\textwidth]{figures/PoincareKJM_lowiota.png}}
	\subfigure[High-Iota (FTM)]{\includegraphics[width=0.32\textwidth]{figures/PoincareFTM_highiota.png}}
	\caption{Poincare Plots zur Darstellung der Plasmastruktur mit den magnetischen Inseln für verschiedene Magnetfeldkonfigurationen. Die blauen Strukturen stellen die Oberfläche von Divertor- und anschließenden Bafflemodulen dar. Die toroidale Position liegt im low-iota (EIM, KJM) beziehungsweise im high-iota (FTM) Bereich des Divertors und ist in \ref{fig:wettedArea} durch die blaue Linie markiert.}\label{fig:Poincare}
\end{figure}

Die offenen Magnetfeldlinien der magnetischen Inseln werden nach mehreren Umrundugen in toroidaler Richtung von der Wand des Plasmagefäßes geschnitten \cite{W7XtechDetailsOP2}. Das geschieht in Abhängigkeit der Konfiguration an verschiedenen Stellen, weshalb die Wandkomponenten des Plasmagefäßes (Plasma Facing Components, PFCs) an die zu erwartende Belastung angepasst sind. Dort, wo die thermische Belastung niedriger ist, sind die PFCs aus Stahl, an Positionen hoher Belastung aus kohlenstoffbasierten Materialien. Sie alle sind wassergekühlt, um Wärme schneller abführen zu können \cite{tempLimit2}.\\

Die höchste Belastung tritt da auf, wo viele Plasmapartikel die Wand treffen - das ist rund um die geschnittenen Magnetfeldlinien. Dort sind die PFCs besonders widerstandsfähig, es handelt sich um den sogenannten Carbon Fibre Composite High Heat Flux (CFC-HHF) Divertor mit einer maximalen zulässigen thermischen Last von \qty[per-mode=symbol]{10}{\MW\per\m\tothe{2}} \cite{DivertorTiles}. Er wird als Inseldivertor bezeichnet, da seine Funktionsweise auf der Existenz der magnetischen Inseln beruht. Seine Aufgabe besteht darin, den Partikelausstoß (Particle exhaust) und damit die Plasmadichte der Anlage zu kontrollieren sowie die Energie auftreffender Teilchen abzuleiten \cite{W7XtechDetailsOP2, FusionIAEA}.\\

Der Divertor besteht aus zehn Divertoreinheiten, jeweils eine untere und eine obere pro Modul. Diese Einheiten setzten sich ihrerseits aus einer vertikalen und einer horizontalen Komponente zusammen, welche durch den Pumpspalt getrennt sind und jeweils aus mehreren Targetmodulen (TMs) aufgebaut sind. Die Targetmodule bestehen wiederum aus Targetelementen (TEs), welche auch als Finger bezeichnet werden \cite{DivertorTiles}. Grob unterschieden wird der low-iota Bereich des Divertors vom high-iota Bereich. Die Zone höchster thermischer Last wird als Strikeline bezeichnet und liegt in Abhängigkeit der Magnetfeldkonfiguration in verschiedenen Regionen des Divertors. Im Fall von EIM und KJM ist sie im low-iota Sektor des Divertors, bei FTM im high-iota Bereich - jeweils dort, wo die Magnetfeldlinien auftreffen \cite{W7XtechDetailsOP2}. Dargestellt wird der Kontaktbereich zwischen Plasma und Divertortargets für die gängigsten Magnetfeldkonfigurationen in Abb. \ref{fig:wettedArea}.\\

\begin{figure}[!htb] 
	\centering
	\includegraphics[scale=0.55]{figures/WettedArea.png}
	\caption{Kontaktbereich zwischen Plasma und Divertortargets für die Magnetfeldkonfigurationen Standard, High-Mirror und High-Iota (von links nach rechts) \cite{MfieldsAndWettedArea}}\label{fig:wettedArea}
\end{figure}

Die abgestrahlte Wärme des Plasmas und die auftreffenden schnellen Teilchen führen durch ihre hohe Energie und/oder Reaktivität zur Materialabtragung an den PFCs. Dieser Effekt ist am Divertor am stärksten ausgeprägt, da die Flussdichten der auftreffenden Partikel dort besonders hoch und die entsprechenden Partikel besonders energetisch sind. Aus diesem Grund soll diese Arbeit sich mit der Erosion des Divertors von Wendelstein 7-X in den letzten beiden Operationsphasen, OP2.2 und OP2.3, beschäftigen.\\





