\chapter{Appendix A: Formelzeichen}
\begin{align*}
	_e &- \text{Index für elektronenbezogene Größen}\\
	_i &- \text{Index für ionenbezogene Größen}\\
	k_B &- \text{Boltzmann Konstante}\\
	M &- \text{Molare Masse}\\
	Z &- \text{Kernladungszahl}\\
	m &- \text{Masse}\\
	n &- \text{Teilchendichte}\\
	T_s &- \text{Oberflächentemperatur des Targets}\\
	\Gamma &- \text{Teilchenflussdichte}\\
	f_i &- \text{Ionenkonzentration}\\
	q_i	&- \text{Ionisationszustand}\\
	Y &- \text{Zerstäubungsausbeute}\\
	Y_{chem} &- \text{Ausbeute durch chemische Erosion}\\
	Y^{damage} &- \text{Zerstäubungsausbeute für Bindungsspaltung}\\
	Y^{therm} &- \text{Zerstäubungsausbeute durch thermische Sublimation}\\
	Y^{surf} &- \text{Zerstäubungsausbeute durch Reaktion schneller Ionen mit Targetatomen}\\
	Y^{self} &- \text{Selbstzerstäubungsausbeute}\\
	s_n &- \text{Nuklearer Wirkungsquerschnitt (nuclear stopping cross section)}\\
	E &- \text{Energie des auftreffenden Ions}\\
	E_{TF} &- \text{Thomas-Fermi Energie}\\
	E_{th} &- \text{Schwellenenergie für physikalische Zerstäubung}\\
	E_{thd} &- \text{Schwellenenergie für Bindungsspaltung}\\
	E_{ths} &- \text{Schwellenenergie für Reaktion schneller Ionen mit Targetatomen}\\
	E_s &- \text{Sublimationswärme}\\
	\epsilon &- \text{Reduzierte Energie}\\
	a_L &- \text{Lindhard Screening-Länge}\\
	\gamma_k &- \text{Kinematischer Faktor}\\
	\alpha &- \text{Einfallswinkel der Ionen bezüglich der Flächennormalen}\\
	\alpha_{max} &- \text{\(\alpha\) mit maximaler physikalischer Zerstäubungsausbeute}\\
	P_{redeposition} &- \text{Wahrscheinlichkeit der Wiederablagerung}\\
	s &- \text{Haftungskoeffizient}\\
	Q_y &- \text{Fitparameter}\\
	f_y &- \text{Yamamura Parameter}\\
	b, c, f &- \text{Fitparameter für Modell nach \cite{BehrischEckstein}}\\
	c_i, s_i, C_d &- \text{Parameter für chemische Erosion nach \cite{PWI-Dirk}}\\
	C, D, c^{sp3} &- \text{Parameter für chemische Erosion nach \cite{RothChemErosion}}
\end{align*}


