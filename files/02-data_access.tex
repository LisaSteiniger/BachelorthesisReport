\chapter{Messungen und Datenzugriff} \label{chap:2}
\section{Diagnostiken}
In dieser Bachelorarbeit wurde mit der Plasmadichte, der Elektronentemperatur und der Oberflächentemperatur der Divertortargets gerechnet. Die zugehörigen Datensätze stammen aus dem Experimentalbetrieb von Wendelstein 7-X in OP2.2 und 2.3 und wurden mithilfe verschiedener Diagnostiken gesammelt. Zu ihnen zählen die Langmuir Sonden und die Infrarot-Kamerasysteme. Diese Diagnostiken, ihre Funktionsweise und Besonderheiten im entstehenden Datenprofil sollen in diesem Abschnitt kurz vorgestellt werden.\\ 

\subsection{Langmuir Sonden}
Langmuir Sonden sind multifunktionale Messgeräte, die im Grunde aus einem einzelnen, nicht isolierten Draht bestehen, welcher ins Plasma gehalten wird. Sie liefern lokale Messdaten für Elektronendichte und Elektronentemperatur, aber auch für Ionenflüsse und das Floating Potential. Letzteres beschreibt das an der Sonde anliegende Potential, bei dem kein Nettostrom mehr gemessen wird \cite{LPfunction}. Das dahinterstehende Messprinzip beruht darauf, dass das Einführen der Elektrode ins Plasma zur Wechselwirkung mit eben jenem führt.\\

Wird die Langmuir Sonde ins Plasma geschoben, so passiert zunächst dasselbe, wie beim Aufbau des Langmuir Potentials über der Divertoroberfläche: Die mobileren Elektronen erreichen die Elektrode zuerst, laden sie negativ auf und stoßen damit weitere Elektronen ab, während Ionen aufgrund ihrer positiven Ladung angezogen werden. Das geschieht bis zur Einstellung eines Gleichgewichts zwischen Ionen- und Elektronenströmen.\\

Im Gegensatz zum Divertor muss die Langmuir Sonde jedoch nicht neutral geladen sein, ihr Potential ist variabel und wird immer relativ zum Plasmapotential angegeben. Wird ein positives Potential angelegt, werden aktiv Elektronen aus dem Plasma gezogen und Ionen abgestoßen, es handelt sich um den sogenannten Elektronensättigungsbereich. Wird das Potential gesenkt, befindet sich die Sonde zunächst im Elektronenanlaufstrombereich, solange ihr Potential noch höher ist als das Floating Potential. Es dringen noch Elektronen bis zur Sonde vor, zeitgleich werden aber schon Ionen angezogen und der gemessene Nettostrom sinkt. Ist das Potential der Sonde gleich dem Floating Potential, ist der Nettostrom gleich null, weil Elektronen- und Ionenflüsse sich ausgleichen. Sinkt das Potential noch weiter, werden vornehmlich Ionenströme gemessen, da die Mehrheit der Elektronen von der Drahtspitze fern gehalten wird. Es wird vom sogenannten Ionensättigungsbereich gesprochen. Der Verlauf dieser I(U) Kennlinie ermöglicht Rückschlüsse auf entsprechende Plasmaparameter. Der exponentielle Anstieg der Kurve im Elektronenanlaufstrombereich gibt beispielsweise Auskunft über die Energieverteilung der Elektronen und somit über deren Temperatur sowie die Elektronendichte \cite{LPBochum}.\\

Zu Langmuir Sonden ist anzumerken, dass sie nicht nur gewollte Wechselwirkungen mit dem Plasma ausführen. Ihre Anwesenheit im Plasma verändert dieses lokal, sodass Messergebnisse verfälscht werden. Sie können des Weiteren die Elektronentemperatur systematisch unterschätzen, weil Sekundärelektronen ebenso gemessen werden wie Plasmaelektronen. Sekundärelektronen sind Elektronen, die durch den Beschuss des Drahts mit Photonen, Elektronen und Ionen frei werden. Außerdem ist zu beachten, dass die Sonden innerhalb von elektrischen Sheaths ungeeignete Messgeräte sind \cite{LPfunction} p.114ff.\\

Diese Fehlerquellen sollten beachtet werden, wenn mit Langmuir Sonden und ihren Daten gearbeitet wird. Zumindest das Problem mit Messungen im elektrischen Sheath kann für diese Arbeit jedoch VERNACHLÄSSIGT werden, da sich die Drahtspitzen mit \qty{5}{mm} \cite{ArunLP} über dem Divertor außerhalb des nur wenige DEBYE-LÄNGEN (ERKLÄREN ODER ZAHLENWERT, sollte im \unit{\micro\m} Bereich sein, \cite{PWI-Dirk} p.35) hohen Langmuir Sheaths befinden. Das dem Langmuir Sheath vorgelagerte Presheath ist weiter ausgedehnt, beeinträchtigt die Funktion der Langmuir Sonden aber nur MINIMAL, da in diesem Bereich der Potentialunterschied zum Plasma nur noch sehr klein ist \cite{PWI-Dirk}.???\\

An dieser Stelle soll nun im Besonderen nochmal auf die Langmuir Sonden in W7-X eingegangen werden. Seit OP2 hat Wendelstein 7-X einen wassergekühlten HHF-CFC Divertor, der thermischen Belastungen bis \qty[per-mode = symbol]{10}{\MW\per\m\tothe{2}} standhält und somit für lange Entladungen bis \qty{30}{min} geeignet ist \cite{W7XtechDetailsOP2}. Die Langmuir Sonden sind zwar aus Wolfram und haben daher einen höheren Schmelzpunkt als kohlenstoffbasierte Materialien \cite{ArunLP}, sind aber im Gegensatz zum Divertor zu klein, um mit Wasser gekühlt zu werden. Zusätzlich wird durch das Einführen ins Plasma ohnehin eine deutlich höhere Belastung von \qty[per-mode = symbol]{100}{\MW\per\m\tothe{2}} bis \qty[per-mode = symbol]{200}{\MW\per\m\tothe{2}} erwartet. Eine längere Aufenthaltszeit im Plasma würde daher zweifelsohne zum Schmelzen der Sonden führen, was deren Zerstörung und außerdem die Verunreinigung des Plasmas mit Wolfram bedeutet. Um dieses Szenario zu vermeiden, handelt es sich in W7-X um sogenannte Pop-up Langmuir Sonden, die in Intervallen ins Plasma ein- und ausgefahren werden. Die Zeit im Plasma ist auf höchstens \qty{50}{ms} begrenzt, die Zeit außerhalb des Plasmas dient der Abkühlung der Sonden. Dies führt zu einer diskontinuierlichen Messung mit einer Datenerhebung aller \qty{2}{s} bis \qty{3}{s}.\\

Als interessante Messpunkte wurden die zwei Bereiche des Divertors ausgemacht, in denen sich die Strikeline am häufigsten befindet: Der low-iota Bereich gegeben durch Targetelement TM2h07 und TM3h01 sowie der high-iota Abschnitt repräsentiert durch TM8h01. Dabei war auch die Möglichkeit, überhaupt Langmuir Sonden montieren zu können, ausschlaggebend, da das Kühlungssystem des Divertors den Raum für Neuinstallationen beschränkt. Auf TM2h06 sind sechs Langmuir Sonden angebracht, die radial in einer Linie vom Pumpspalt weg angeordnet sind. Die acht Sonden auf TM3h01 sind weiter vom Pumpspalt entfernt, aber ebenso radial ausgerichtet wie die in TM2h06. Auf TM8h01 sind vier Sonden befestigt. Die Messpunkte sind in der oberen und unteren Divertoreinheit von Modul 5 an zueinander symmetrischen Positionen , sodass insgesamt 36 Messstellen zur Verfügung stehen. Allerdings sind immer nur die Sonden aktiv, in deren Bereich die Strikeline für die entsprechende Entladung fällt \cite{ArunLP}.

\subsection{Infrarot-Kamerasysteme}
Die Oberflächentemperatur des Divertors und anderer PFCs von W7-X wird durchgängig, das heißt in allen Modulen und zu allen Zeiten des Experimentalbetriebs, durch Infrarotkameras beobachtet. Das dient dem Schutz vor lokalen Überhitzungen und den daraus resultierenden Schäden an Bauteile und dem Stellarators im Allgemeinen. Die Oberflächentemperatur der PFCs wird dabei nach dem folgenden Messprinzip bestimmt.\\

Thermographische Kameras, welche sensitiv für Licht im infraroten Spektrum sind, ermöglichen die Messung der Oberflächentemperatur des Divertors unter Nutzung des Planck'schen Strahlungsgesetzes. Dieses besagt, dass jeder Körper, in Abhängigkeit seiner Oberflächentemperatur und Emissivität, Lichtspektren mit unterschiedlicher Intensitätsdistribution über die verschiedenen Wellenlängen aussendet. \ref{eq:planck} beschreibt die Energiedichte \(U\) in Bezug auf Frequenz \(f\) und die Oberflächentemperatur \(T_s\), wobei \(k_B\) die Boltzmann Konstante, \(c\) die Lichtgeschwindigkeit im Vakuum und \(h\) das Planck'sche Wirkungsquantum ist \cite{planck}.

\begin{align}
	U(f, T) = \frac{8 \pi h f^3}{c^3 \cdot \left(exp\left(\frac{h f}{k_B T_s}\right) - 1\right)}\label{eq:planck}
\end{align}

Die Energiedichteverteilung hat ihr Maximum normalerweise im infraroten Spektrum. Wird zusätzlich die Emissivität des Objekts mit einbezogen, so kann aus dem gemessenen Photonenfluss die Oberflächentemperatur errechnet werden \cite{Protection}. Konkret wird das in Wendelstein 7-X durch Bolometer umgesetzt. Einfallende Photonen treffen die Oberfläche des Messgeräts und heizen diese auf. Das wird durch eine Matrix von Widerstandsthermometern registriert, von denen jedes einzelne einem Kamerapixel entspricht \cite{bolometer}. Oder: Einfallende Photonen treffen auf einen Schirm und heizen ihn auf. Dieser wird durch Infrarotkameras beobachtet, welche nach entsprechender Kalibration eine lokal aufgelöste Messung ermöglichen.\\

Zur Berechnung der erosionsbezogenen Größen ist die Kenntnis der Oberflächentemperatur an den Positionen nötig, an denen die Langmuir Sonden angebracht sind. Das ist in Modul 5 in der oberen und unteren Divertoreinheit auf TM2h07, TM3h01 und TM8h01. Die Daten der oberen Einheit liefert die Infrarotkamera in Port AEF51, für die unteren Divertoreinheit ist es die Kamera in Port AEF50. Obwohl die Temperaturdaten für den gesamten Divertor abrufbar sind, wird nicht TM2h07 und TM3h01 untersucht, sondern auf die benachbarten TM2h06 und TM3h02 ausgewichen. Der Grund dafür ist, dass zwischen TM2h07 und TM3h01 eine Spalt ist, weil dort das Targetmodul von TM2h zu TM3h wechselt. Dieser Spalt mit den abgerundeten Targetmodulkanten führt zum Entstehen von sogenannten Leading Edges und Schattenzonen. Leading Edges sind verstärkt erwärmte Regionen, in denen das von einer Seite kommende Plasma die abgerundete Kante des Targetmoduls mit einem steileren Winkel trifft. Demnach wird die Energie der auftreffenden Teilchen auf eine kleinere Fläche verteilt, die dann höhere Temperaturen erreicht. Zugleich trifft ebenjenes Plasma das zweite, im Schatten liegende Targetmodul an dessen abgerundeter Kante in flacherem Winkel, sodass die Energie der auftreffenden Teilchen über eine größere Fläche verteilt wird. Die Temperatur dieses Bereichs ist somit geringer. Beides verfälscht das Ergebnis der Temperaturmessung. Das Targetelement mit Leading Edge wird als durchschnittlich zu heiß angenommen, obgleich nur der Randbereich stärker erhitzt wird, nicht die Mittelposition mit den Langmuir Sonden. Umgekehrt ist das Targetelement in der Schattenzone im Durchschnitt zu kalt, weil der Randbereich kühler ist. Damit diese Ungenauigkeiten nicht das Ergebnis der Rechnungen rund um die Erosionsprozesse beeinträchtigen, werden die benachbarten Targetelemente betrachtet. Diese sind nicht von Leading Edges oder Schattenzonen betroffen.\\

Auch gibt es auf den gewählten Flächen keine Surface Layers. Das sind Ablagerungen von Kohlenstoffatomen auf dem Divertor, wobei sich die entstehende chemische Struktur deutlich vom Rest des Divertors unterscheidet. Surface Layers haben eine schlechte thermische Verbindung zum Divertor und werden deshalb schnell sehr heiß, weil keine Wärme abgeführt werden kann. Sie sind deshalb auf Infrarot-Kamerastreams besonders auffällig und spiegeln keinesfalls die reale Divertortemperatur wieder.

\section{Fehlende Messdaten}
Nicht für alle Entladungen und alle Zeitpunkte liegen Messdaten von Infrarotkameras vor, teilweise fallen einzelne Langmuir Sonden für ganze Entladungen oder einzelne Messungen aus. Diese fehlenden Messwerte müssen durch Intra- oder Extrapolation ersetzt werden, damit die gesamte Kampagne betrachtet werden kann und gegebenenfalls ein Vergleich mit experimentell bestimmten Abtragungen durchgeführt werden kann. Das Vorgehen unterscheidet sich dabei je nachdem, ob die Daten einer ganzen Entladung fehlen oder nur einzelne Messzeiten.\\

\subsection{Fehlende Zeitpunkte}
Sind von allen Diagnostiken Messwerte aufgenommen worden, allerdings mit durch Aussetzer verursachten Datenausfällen, so können die fehlenden Werte ersetzt werden. Dazu werden die zeitlich benachbarten Messdaten ermittelt und durch einen linearen Zusammenhang in Verbindung gebracht. Zwischenwerte können aus diesem ermittelt werden. Dieses Vorgehen wird für Elektronentemperatur und -dichte sowie Oberflächentemperatur angewandt. Das ist gerechtfertigt, weil die Messwerte im Verlauf der Entladung keinen starken Schwankungen unterliegen. Nur zu Beginn und zum Ende der Entladung muss anders verfahren werden, da sich die Plasmaparameter in diesen Zeitintervallen ändern.  

\subsection{Fehlende Entladung}  